% Created 2019-09-23 Mon 23:21
% Intended LaTeX compiler: pdflatex
\documentclass[10pt,t]{beamer}
\usepackage[utf8]{inputenc}
\usepackage[T1]{fontenc}
\usepackage{graphicx}
\usepackage{grffile}
\usepackage{longtable}
\usepackage{wrapfig}
\usepackage{rotating}
\usepackage{amsmath}
\usepackage{textcomp}
\usepackage{amssymb}
\usepackage{capt-of}
\usepackage{hyperref}
\usetheme{default}
\author{L. Larrabee Strow}
\date{\today}
\title{\large AIRS Calibration for Climate Trend Studies: Status and Future}
\subtitle{\footnotesize{AIRS Science Team Meeting}}
\date{\vspace{0.1in}\footnotesize{September 25, 2019\vfill}}
\author{L. Larrabee Strow\inst{1,2}}
\institute[UMBC]{\inst{1} UMBC Physics Dept. \and \inst{2}UMBC JCET}
\input beamer_setup
\usetheme{metropolis}
\metroset{titleformat title=allcaps}
\renewcommand{\UrlFont}{\small\tt}
\renewcommand*{\UrlFont}{\footnotesize}
\tolerance=1000
\RequirePackage{fancyvrb}
\DefineVerbatimEnvironment{verbatim}{Verbatim}{fontsize=\footnotesize}
\begin{document}

\maketitle
\addtobeamertemplate{block begin}{
  \setlength{\parsep}{0pt}
  \setlength{\topsep}{3pt plus 2pt minus 2.5pt}
  \setlength{\itemsep}{0pt plus 0pt minus 2pt}
  \setlength{\partopsep}{2pt}
}


\begin{frame}[label={sec:orge0f03c3},shrink=30]{Calibration Requirements for Climate Science}
\vspace{-0.1in}
\begin{large}
\begin{itemize}
\item AIRS 17+ year record long enough to address key climate questions
\item Stability of radiometric calibration is key
\item AIRS sensitivity to \cd, SST, etc allows stringent tests of stability
\end{itemize}
\end{large}
\vspace{-0.2in}
\begin{columns}
\begin{column}{0.55\columnwidth}
\begin{block}{Climate Science Questions}
\vspace{0.05in}
\emph{All require min \textasciitilde{}0.1K/decade stability}
\vspace{-0.05in}
\begin{itemize}
\item Global Trending: T(z), \water(z), T\textsubscript{surf}
\item Water vapor feedback (Does relative humidity vary)
\item Cloud feedback
\item Trends in PBL cloud occurrence
\item OLR anomalies separated by cause: T/\water/cloud/surface, etc.
\end{itemize}
\end{block}
\end{column}

\begin{column}{0.55\columnwidth}
\begin{block}{Hyperspectral IR Advantages}
\begin{itemize}
\item AIRS senses both climate forcings, and responses
\item Clean separation of tropospheric vs stratospheric temperature trends (unlike microwave)
\item Multiple long-term overlapping missions (AIRS, CrIS, IASI)
\item AIRS, CrIS, IASI already agree to \textasciitilde{}0.1-0.3K and can be merged to 0.03K or better.
\end{itemize}
\end{block}
\end{column}
\end{columns}



\vspace{0.2in}
\begin{large}
Significant AIRS calibration drifts have \textbf{already} resulted publication of in-accurate data that were publicized by NASA/GSFC and the media (Washington Post, Scientific American).  \emph{This talk suggests how to make AIRS an accurate instrument for climate science.}
\end{large}
\end{frame}
\end{document}