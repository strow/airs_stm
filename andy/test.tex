\documentclass[10pt,t]{beamer}
\usepackage[utf8]{inputenc}
\usepackage[T1]{fontenc}
\usepackage{graphicx}
\usepackage{grffile}
\usepackage{longtable}
\usepackage{wrapfig}
\usepackage{rotating}
\usepackage{amsmath}
\usepackage{textcomp}
\usepackage{amssymb}
\usepackage{capt-of}
\usepackage{hyperref}
\usetheme{default}
%---------------------------------------------------------------------
\title{Statistical Approaches for Simple Measurements of  Surface Temperature and
Cloud Forcing Trends and Extrema} 
\author{Andrew Tangborn , L. Larrabee Strow and \\ Howard Motteler }
\institute{Joint Center for Earth Systems Technology and \\ UMBC Department of Physics}
\date{AIRS STM  -- Sept. 26, 2019}
%---------------------------------------------------------------------
\input beamer_setup
\usetheme{metropolis}
\metroset{titleformat title=allcaps}
\renewcommand{\UrlFont}{\small\tt}
\renewcommand*{\UrlFont}{\footnotesize}
\tolerance=1000
\RequirePackage{fancyvrb}
\DefineVerbatimEnvironment{verbatim}{Verbatim}{fontsize=\footnotesize}
\begin{document}

\maketitle
\addtobeamertemplate{block begin}{
  \setlength{\parsep}{0pt}
  \setlength{\topsep}{3pt plus 2pt minus 2.5pt}
  \setlength{\itemsep}{0pt plus 0pt minus 2pt}
  \setlength{\partopsep}{2pt}
}
%---------------------------------------------------------------------
%---------------------------------------------------------------------
%---------------------------------------------------------------------
\frame{\maketitle}
%---------------------------------------------------------------------
%---------------------------------------------------------------------
%---------------------------------------------------------------------
%---------------------------------------------------------------------
\begin{frame}
  \frametitle{Overview}
  % \begin{itemize}
  % \item Larrabee
    \begin{itemize}
    \item AIRS has made nearly 17 years of high quality TOA radiance measurements 
    \item We have previously shown that the instrument stability is sufficient to determine linear rates surface temp., column $CO_2$, temp. and wv profiles  
    \item We have also shown that probability density functions (PDFs) of clear sky PDFs can provide insight into non-Gaussian climate variability and stochastic forcing of the atmosphere 
    \item We want to learn whether the 16+ years is long enough to start to see climate signals in surface temperature and cloud forcing.   
    \item The primary issue to resolve is whether the rates we calculate have a significant contribution from the ENSO cycle.  
    \end{itemize}
\end{frame}
%---------------------------------------------------------------------
\begin{frame}
  \frametitle{The ENSO cycle during the past 10 years}
  \framesubtitle{ Southern oscillation index (SOI)} 

The SOI is a measure of the pressure difference between Tahiti and Darwin:
$$ 
SOI =  \frac{\Delta P - \Delta P_{average}}{\sigma_{\Delta P}}
$$

%\includegraphics[width=1.0\linewidth]{./Figs/combined_pdf.png}

\end{frame}

\end{document}
% %---------------------------------------------------------------------------

         
% \begin{frame}
%   \frametitle{Mean PDF from 13 years of allsky observations from 1231 \wn, all latitudes.}
%   \framesubtitle{Window channel sensitive to surface temperature, clouds and column water vapor}
%   \vspace{-0.15in}
%   \begin{itemize}
%   \item PDF scale is indicated by the colorbar. The x and y axes show the latitude and BT bins. 
%   \end{itemize}
%   \vspace{0.1in}
%   \centering
% %  \includegraphics[width=0.7\linewidth]{./Figs/bt1291_vs_lat_abs_pdf_with_stemp.png}
% %   \includegraphics[width=0.7\linewidth]{./Figs/bt1231_day_global_absolute_pdf_obs_720.png}
% \end{frame}
% %---------------------------------------------------------------------
% %---------------------------------------------------------------------
% \begin{frame}
%  \frametitle{Rate of change in BT PDF for $1231 \wn$}
% \begin{itemize}
%    \item Linear rate from regression of $1231 \wn$ PDF.
%    \item PDF rate shows how occurences of a particular BT range are changing per year.
%    \item Color bar scale shows whether BT is increasing. 
%    \item Gray lines are where rate < uncertainty 
%    \includegraphics[width=0.65\linewidth]{./figures/bt1291_vs_lat_abs_pdf_rate.png}
% \end{itemize}
% \end{frame}
% %---------------------------------------------------------------------
% \begin{frame}
%  \frametitle{Cloud radiative effect}
% \begin{itemize}
%    \item Mean cloud forcing over 13 years.
%    \item Clear Calculated Bt - Obs using ERA.
%    \item Averaged on 1x0.5 degree grid. 
% \end{itemize}
% %   \includegraphics[width=0.7\linewidth]{./Figs/map_mean_forcing.png}
% \end{frame}
% %---------------------------------------------------------------------
% \begin{frame}
%  \frametitle{Observation count for cloud forcing ranges}
% \begin{itemize}
%   \item Observations per pixel.
%   \item 1\% of data over 13 years.
% \end{itemize}
% \twocol{.55}{.55}
% {
% \begin{block}{\tiny 1-5 K }
% %\includegraphics[width=\linewidth]{./Figs/map_mean_forcing_1to5K.png}
% \end{block}
% }
% {
% \begin{block}{\tiny 5-15 K}
% %\includegraphics[width=\linewidth]{./Figs/map_mean_forcing_5to15K.png}
% \end{block}
% }
% \end{frame}
% %---------------------------------------------------------------------
% \begin{frame}
%  \frametitle{Observation count - continued }
% \begin{itemize}
%    \item Observations per pixel.
%    \item 1\% of data over 13 years.
% \end{itemize}
% \twocol{.55}{.55}
% {
% \begin{block}{\tiny 15-30 K }
% %\includegraphics[width=\linewidth]{./Figs/map_mean_forcing_15to30K.png}
% \end{block}
% }
% {
% \begin{block}{\tiny 30-45 K}

% %\includegraphics[width=\linewidth]{./Figs/map_mean_forcing_30to45K.png}
% \end{block}
% }
% \end{frame}

% %---------------------------------------------------------------------
% \begin{frame}
%  \frametitle{Mean Cloud Forcing over 13 years, zonal average}
% \begin{itemize} 
%   \item Color bar indicates PDF value.
%   \item Large values indicate deep convective clouds. 
%   \item Values near zero indicate clear sky.  
% \end{itemize} 
% %    \includegraphics[width=0.7\linewidth]{./Figs/fig5_pdf_ch1291_global_night.png}
% \end{frame}

% %---------------------------------------------------------------------
% \begin{frame}
% \frametitle{Percent rate of change in cloud forcing ($1231 \wn$). }

% \begin{itemize}
%   \item Rates form linear regression of cloud forcing over 13 yrs
%   \item Color scale indicates percent change in PDF per year
%   \item Regions with dots have uncertainty greater than rate
% \end{itemize}
% %    \includegraphics[width=0.7\linewidth]{./Figs/fig2_pdf_ch1291_global_night_obs.png}
% \end{frame}

% %---------------------------------------------------------------------
% \begin{frame}
% \frametitle{Mean Total Cloud Fraction over 13 years}
% \begin{itemize} 
%    \item Sum of cloud forcing PDFs from 5K to maximum 
% \end{itemize} 
% %   \includegraphics[width=0.8\linewidth]{./Figs/fig3_pdf_ch1291_global_night_obs.png}
% \end{frame}
% %-----------------------------------------------------------------------
% \begin{frame}
%   \frametitle{Percent rate in change of cloud fraction}
%   \begin{itemize}
%   \item Sum rate of change in cloud forcing from 5 K to maximum  
%   \item Uncertainty from linear regression  
%   \end{itemize}
% %   \includegraphics[width=0.7\linewidth]{./Figs/fig4_pdf_ch1291_global_night_obs.png}
% \end{frame}

% %---------------------------------------------------------------------
% \begin{frame}
% \frametitle{Imager determination of cloud fraction}
% \vspace{-0.2in}
% \begin{footnotesize}
% \begin{itemize}
%    \item ISCCP = count how many cloudy 5 km pixels there are in a 280 km region, seen by satellite? http://isccp.giss.nasa.gov/cloudtypes.html
%    \item PATMOS : from AVHRR, cf from tests using IR/NR/VIS channels  (Foster et. al., Remote Sens. 2016, 8(5), 424; doi:10.3390/rs8050424)
% \end{itemize}
% \end{footnotesize}
% \twocol{0.55}{0.55}
% {
% \begin{block}{\tiny Norris (Nature 2016)}
% %\includegraphics[width=\linewidth]{./Figs/strow_1.pdf}
% \end{block}
% }
% {
% \begin{block}{\tiny Stowe (JGR 1997)}
% %\includegraphics[width=\linewidth]{./Figs/strow_2.png}
% \end{block}
% }
% \end{frame}
% %-----------------------------------------------------------------------
% \begin{frame}
%   \frametitle{Reduction of Uncertainties over Time}
%   \begin{itemize} 
%     \item Linear regression errors decrease with longer data sets 
%     \item We calculate the regression errors for 1-13 year linear fits and extrapolate to 25 years
%   \end{itemize}
% \twocol{.55}{.55}
% {
% \begin{block}{\tiny 25 years}
% %\includegraphics[width=\linewidth]{./Figs/cloud_percent_error_forecast.png}  
% \end{block}
% }
% {
% \begin{block}{\tiny Zoom to last 15 years}

% %\includegraphics[width=\linewidth]{./Figs/cloud_percent_error_forecast_zoom.png}  
% \end{block}
% }
% \end{frame}
% %-----------------------------------------------------------------
% \begin{frame}
% \frametitle{Conclusions} 
% \begin{itemize}
%    \item    AIRS can give you high-quality pseudo-veritcal cloud percent changes, and we are starting to reach climate level measurements 
%    \item   Linear rates of cloud forcing can be used to obtain cloud fraction rate dependence on latitude and level 
%    \item   Longer term IR observation records should lead to higher accuracy 
%    \item   Introduced a simple and easily implemented  definition of partial cloudiness that agrees with eg ISCCP and PATMOS
%    \item   Preliminary results, more work needed 
% \end{itemize}

% \end{frame}


% \begin{frame}
%   \frametitle{Cloud Forcing PDF with ERA Calcs, Ocean Only}
%   \begin{itemize}
%   \item  Shows much less negative forcing over oceans 
%   \end{itemize}
%    \includegraphics[width=0.8\linewidth]{./figures/forcing_oceans_1291.png}
% \end{frame}

% %-----------------------------------------------------------------------

% \begin{frame}
%   \frametitle{Cloud forcing PDF at equator, ERA Calcs, one month}
%   \begin{itemize}
%   \item  Sum of these PDF values gives the cloud fraction. 
%   \end{itemize}
%    \includegraphics[width=0.8\linewidth]{./figures/cloud_forcing_1291_equator_ocean.png}
% \end{frame}
% %-----------------------------------------------------------------------

% \begin{frame}
%   \frametitle{Cloud forcing rates, ERA Calcs, ocean only}
%   \begin{itemize}
%     \item Shows high clouds increasing over tropics and northern mid-lat.
%   \end{itemize}
%    \includegraphics[width=0.8\linewidth]{./figures/over_ocean_cloud_forcing_rates_1291.png}
% \end{frame}
% %-----------------------------------------------------------------------

% \end{document}
% %-----------------------------------------------------------------------
% %-----------------------------------------------------------------------


% %%% Local Variables:
% %%% mode: latex
% %%% TeX-master: t
% %%% End:
