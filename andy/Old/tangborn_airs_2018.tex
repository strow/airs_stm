\documentclass{beamer}
%
% Choose how your presentation looks.
%
% For more themes, color themes and font themes, see:
% http://deic.uab.es/~iblanes/beamer_gallery/index_by_theme.html
%
\mode<presentation>
{
  \usetheme{default}      % or try Darmstadt, Madrid, Warsaw, ...
  \usecolortheme{default} % or try albatross, beaver, crane, ...
  \usefonttheme{default}  % or try serif, structurebold, ...
  \setbeamertemplate{navigation symbols}{}
  \setbeamertemplate{caption}[numbered]
} 

\usepackage[english]{babel}
\usepackage[utf8x]{inputenc}

%\title[Wavelets]{Wavelet Transforms in Time Series Analysis}
%\author{Andrew Tangborn}
%\institute{Planetary Geodynamics Laboratory, Goddard Space Flight Center}
%\date{May 2, 2018}

\begin{document}

\begin{frame}
  \titlepage
\end{frame}

% Uncomment these lines for an automatically generated outline.
%\begin{frame}{Outline}
%  \tableofcontents
%\end{frame}

\section{Outline}

\begin{frame}{Outline}

\begin{itemize}
  \item Fourier Transforms 
  \item What is a Wavelet.
  \item Continuous and Discrete Wavelet Transforms.
  \item Construction of Wavelets through dilation equations.
  \item Example: Haar wavelets.
  \item Daubechies Compactly Supported Wavelets
  \item Data compression, efficient representation
  \item Soft Thresholding
  \item Continuous Transform - Morlet Wavelet
  \item Applications to approximating error correlations
\end{itemize}

\end{frame}

\begin{frame}{Fourier Transforms}
$\bullet$ A good way to understand how wavelets work and why they are useful is by comparing them with Fourier Transforms.
\vskip .01in
$\bullet$ The Fourier Transform converts a time series into the frequency domain:
\vskip .01in
\noindent
\underline{Continuous Transform} of a function f(x):
\vskip .01in
\[ \hat{f}(\omega) = \int\limits^{\infty}_{-\infty} f(x)  e^{-i \omega x} dx \]
where $\hat{f} (\omega)$ represents the {\it strength} of the function at frequency $\omega$,
where $\omega$ is continuous.
\end{frame}

\begin{frame}{Fourier Transforms (continued)}
\underline{Discrete Transform} of a function f(x):
\vskip .01in
\[ \hat{f}(k) = \int\limits^{\infty}_{-\infty} f(x) e^{-ikx} dx \]
where $k$ is a discrete number.
\vskip .01in
\noindent
for discrete data $f(x_j)$, $j=1,...,N$
\[ \hat{f}_k = \sum^N_{j=1} f_j e^{(-i2\pi (k-1)(j-1)/N)} \]
\vskip .01in
\noindent
$\bullet$ The Fast Fourier Transform (FFT) is $o(NlogN)$ operations.

\end{frame} 

\begin{frame}{Example: Single Frequency Signal} 

\[ f(t) = \sin(2\pi t) \]
\vskip .1in
\begin{figure}[ht]
\begin{centering}
\begin{tabular}{cc}
\includegraphics[width=0.4\textwidth]{figures/single_freq.ps} & 
\includegraphics[width=0.4\textwidth]{figures/coef_single_four.ps}\\
{\bf Signal} & 
{\bf Fourier Transform - note the complex values.}\\
\end{tabular}
\end{centering}
\end{figure}
\end{frame} 

\begin{frame}{Two Frequency Signal} 

\[ f(t) = \sin(2\pi t) + 2 \sin(4\pi t) \]
\vskip .1in
\begin{figure}[ht]
\begin{centering}
\begin{tabular}{cc}
\includegraphics[width=0.4\textwidth]{figures/two_freq.ps} & 
\includegraphics[width=0.4\textwidth]{figures/coef_2frea.ps} \\
{\bf Signal} & 
{\bf Fourier Transform}\\
\end{tabular}
\end{centering}
\end{figure}
\vskip .01in 
\noindent
$\bullet$ Both signal frequencies are represented by the Fourier Coefficients.

\end{frame} 


\begin{frame}{Intermittent Signal} 

\noindent
$  f(t)= \sin(2\pi t) + 2 \sin(4\pi t) $ when $.37 < t < .55 $
\vskip .02in
\noindent
$  f(t) = \sin(2\pi t) $ otherwise
\vskip .05in

\begin{centering}
\begin{figure}[ht]
\begin{tabular}{cc}
\includegraphics[width=0.4\textwidth]{figures/two_freq_partial.ps} & 
\includegraphics[width=0.4\textwidth]{figures/two_freq_partial_coef.ps} \\
{\bf Signal} & {\bf Fourier Transform}\\
\end{tabular}
\end{figure}
\end{centering}
\vskip .05in
\noindent
$\bullet$ First frequency is found, but higher intermittent frequency appears
as many frequencies, and is not clearly identified.

\end{frame} 

\begin{frame}{What is a Wavelet?} 
\noindent
$\bullet$ A function that is localized in time and frequency, generally with a zero mean.
\vskip .05in
\noindent
$\bullet$ It is also a tool for decomposing a signal by location {\it and} frequency.
\vskip .02in
\noindent 
Consider the Fourier transform:
\vskip .01in
$\bullet$ A signal is only decomposed into its frequency components.
What happens when applying a Fourier transform to a signal that has a time varying frequency?
\vskip .01in 
\begin{centering}
\begin{figure}[htp]
\includegraphics[width=0.28\textwidth]{figures/var_freq.eps} \\ 
\end{figure}
\end{centering}
\noindent
 $\bullet$ The Fourier transform will only give some information on
 which frequencies are present, but will give no information on when they occur.
\end{frame} 

% Commands to include a figure:
%\begin{figure}
%\includegraphics[width=\textwidth]{your-figure's-file-name}
%\caption{\label{fig:your-figure}Caption goes here.}
%\end{figure}


\begin{frame}{Schematic Representation of Decompositions} 

\noindent
{\bf Original Signal}
\begin{centering}
\begin{figure}[htp]
\includegraphics[width=0.26\textwidth]{figures/signal_decomp.eps} \\
\end{figure}
\end{centering}
\noindent
$\bullet$ The signal is represented by an amplitude that is changing in time.
\noindent
$\bullet$ There is no explicit information on frequency.
\noindent
{\bf Fourier Transform}
\begin{centering}
\begin{figure}[htp]
\includegraphics[width=0.3\textwidth]{figures/fourier_decomp.eps} \\ 
\end{figure}
\end{centering}
\noindent
$\bullet$ The Fourier transform results in a representation that depends only on frequency, but no information on time.

\end{frame} 


\begin{frame}{Wavelet Transform} 

\begin{centering}
\begin{figure}[htp]
\includegraphics[width=0.4\textwidth]{figures/wavelet_decomp.eps} \\ 
\end{figure}
\end{centering}
\vskip .01in
\noindent
$\bullet$ The wavelet transform contains information on both the time location and frequency of a signal.
\vskip .01in
\noindent
{\bf Some typical (but not required)  properties of wavelets}
\vskip .01in
\noindent
$\bullet$ Orthogonality - Both wavelet transform matrix and wavelet functions can be orthogonal.
\vskip .01in
$\bullet$ Useful for creating basis functions for computation.
\vskip .01in
\noindent
$\bullet$ Zero Mean (admissibility condition) - Forces wavelet functions to {\it wiggle} (oscillate between positive and negative).
\vskip .01in
\noindent
$\bullet$ Compact Support - Efficient at representing localized data and functions.

\end{frame} 


\begin{frame}{How are wavelets defined?}

{\bf $\bullet$ Families of basis functions that are based on}
\vskip .02in
$ \bullet$ (1) dilations:
\[ \psi(x) \rightarrow \psi(2x) \]
\vskip .02in
$ \bullet$ (2) translations:
\[ \psi(x) \rightarrow \psi(x+1) \]
\vskip .02in
of a given general "mother wavelet"  $\psi(x)$.
\vskip .02in
\noindent
{\bf $\bullet$ How do we use $\psi(x)$?}
\vskip .01in
$ \bullet$  The general form is:
\vskip .01in
\[ \psi_k^j(x) = 2^{j/2}\psi(2^jx-k) \]
\vskip .01in
where
\vskip .01in
$ \bullet$ j: dilation index
\vskip .01in
$ \bullet$ k: translation index
\vskip .01in
$ \bullet$ $2^{j/2}$ needed for normalization
\vskip .01in
\noindent
{ $\bullet$ How do we get $\psi(x)$?}
\vskip .01in
$\bullet$ Dilation Equations

\end{frame} 


\begin{frame}{Construction of Wavelets}

\noindent
{\bf $\bullet$  We consider here only orthogonally/compactly supported wavelets}
\vskip .01in
 - Orthogonality means:
\[ \int\limits^{\infty}_{-\infty} \psi^j_k(x)\psi^{j'}_{k'}(x)dx = \delta_{kk'}\delta_{jj'} \]
\vskip .02in
\noindent
{\bf $\bullet$ Wavelets are constructed from scaling functions, $\phi(x)$ :}
$\phi(x)$ come from the dilation equation:
\vskip .01in
\[ \phi(x) = \sum_k c_k \phi(2x-k) \]
\vskip .01in
\noindent
$c_k$: {\bf Finite} set of filter coefficients
\vskip 0.01in
\noindent
{\bf $\bullet$ General features:}
\vskip .01in
- Fewer non-zero $c_k$'s mean {\bf more} compact and {\bf less} smooth functions
\vskip .01in
- More non-zero $c_k$'s mean {\bf less} compact and {\bf more} smooth functions 

\end{frame} 

\begin{frame}{Restrictions on Filter Coefficients: Normality}
\[ \int\limits^{\infty}_{-\infty} \phi(x)dx = 1 \]
\vskip .001in
\[ \int\limits^{\infty}_{-\infty} \sum_k c_k \phi(2x-k)dx = 1 \]
\vskip .001in
\[ \sum_k c_k \int\limits^{\infty}_{-\infty} \phi(2x-k)dx = 1 \]
\vskip .001in
\[ \sum_k c_k \int\limits^{\infty}_{-\infty} \frac{1}{2} \phi(2x-k) d(2x-k) = 1 \]
\vskip .001in
\[ \sum_k c_k \int\limits^{\infty}_{-\infty} \frac{1}{2} \phi(\xi) d(\xi) = 1 \]
\vskip .001in
$ \sum_k c_k (\frac{1}{2}) (1) = 1 $ 
or
$ \sum_k c_k = 2 $

\end{frame} 


\begin{frame}{Simple Examples} 

\vskip .01in
 - Smallest number of $c_k$ is 1: just get $\phi(x)=\delta$, $\rightarrow$ zero support.
\vskip .01in
- Haar Scaling Function: $c_0=1$, $c_1=1$.
\vskip .01in 
The scaling function equation is:
\[ \phi(x) = \phi(2x) + \phi(2x-1) \]
\vskip .1in
\noindent
The only function that satisfies this is:
\vskip .1in
\noindent
$ \phi(x) = 1$ if $0 \leq x \leq 1$
\vskip .02in
\noindent
$ \phi(x) = 0 $ otherwise
\begin{centering}
\begin{figure}[ht]
\includegraphics[width=0.4\textwidth]{figures/haar_sf.eps} \\ 
\end{figure}
\end{centering}

\end{frame} 


\begin{frame}{Haar Wavelet} 

{\bf $\bullet$ Wavelets are constructed by taking differences of scaling functions}
\vskip .02in
\[ \psi(x) = \sum_k (-1)^k c_{1-k} \phi(2x-k) \]
\vskip .02in
differencing is caused by the $(-1)^k$:
\vskip .02in
so the basic Haar wavelet is:
\begin{centering}
\begin{figure}[ht]
\includegraphics[width=.3\textwidth]{figures/haar_mother.eps}
\end{figure}
\end{centering}
\end{frame}
\begin{frame}{Family of Haar wavelets}
\vskip .1in 
\noindent
The family comes from dilating and translating:
\[ \psi^j_k(x) = 2^{j/2} \psi(2^jx - k) \]
\vskip .01in
\noindent
so that the $j=1$ wavelets are:
\begin{centering}
\begin{figure}[ht]
\includegraphics[width=.3\textwidth]{figures/haar_j1.eps} \\ 
\end{figure}
\end{centering}
\end{frame} 

\begin{frame}{Orthogonality of Haar wavelets}

{\bf $\bullet$ Translation $\Rightarrow$ no overlap}
\begin{centering}
\begin{figure}[ht]
\includegraphics[width=.3\textwidth]{figures/haar_j1.eps} \\
\end{figure}
\end{centering}

{\bf $\bullet$ Dilation $\Rightarrow$ cancellation}
\begin{centering}
\begin{figure}[ht]
\includegraphics[width=.3\textwidth]{figures/haar_orthog.eps} \\
\end{figure}
\end{centering}

\end{frame} 


\begin{frame}{Daubechies' Compactly Supported Wavelets}

The following plots are wavelets created using larger numbers of filter
coefficients, but having all the properties of orthogonal wavelets. For example,
$D_4$ has coefficients:
\[ c_k = \frac{1}{4}(1+\sqrt{3}), \frac{1}{4} (3 + \sqrt{3}) , \frac{1}{4} (3 - \sqrt{3}), \frac{1}{4}(1 - \sqrt{3})\]
(Reference: Daubechies, 1988)
\begin{centering}
\begin{figure}[ht]
\begin{tabular}{cc}
\includegraphics[width=.25\textwidth]{figures/d4.ps} & 
\includegraphics[width=.25\textwidth]{figures/d12.ps} \\
{\bf $D_4$} & {\bf $D_{12}$}\\
\includegraphics[width=.25\textwidth]{figures/d20.ps} \\ 
{\bf $D_{20}$} \\
\end{tabular}
\end{figure}
\end{centering}

\end{frame} 

\begin{frame}{Continuous and Discrete Wavelet Transforms}

\underline{Continuous}
\vskip .01in 
\noindent 
\[ (T^{wav} f)(a,b) = \mathopen| a \mathclose|^{-1/2} \int dt f(t) \psi(\frac{t-1}{b}) \]
\vskip .02in
\noindent
where
\vskip .02in
\noindent
 a: translation parameter
\vskip .02in
\noindent
 b: dilation parameter
\vskip .02in
\noindent
\underline{Discrete}
\vskip .02in
\[ T^{wav}_{m,n}(f) = a_o^{-m/2}\int dt f(t) \psi(a_o^{-m} t - nb_o) \]
\vskip .02in
\noindent
m: dilation parameter
\vskip .02in
\noindent
n: translation parameter
\vskip .02in
\noindent
$a_o$,$b_o$ depend on the wavelet used

\end{frame} 

\begin{frame}{Fast Wavelet Transform} 
\begin{centering} 
(Reference: S. Mallat, 1989) \\
\end{centering}
\normalsize
\vskip .02in
{\bf $\bullet$} Uses the discrete data:
\[ \left[ \begin{array}{cccccccc} 
f_0 &
f_1 &
f_2 & 
f_3 &
f_4 & 
f_5 & 
f_6 & 
f_7 \\ 
\end{array} \right]
\]
\vskip .03in
{\bf $\bullet$ Pyramid Algorithm $\Rightarrow$ o(N) !! }
\begin{centering}
\begin{figure}[ht]
\begin{tabular}{c}
\includegraphics[width=.6\textwidth]{figures/mallat_2009.eps} \\
\end{tabular}
\end{figure} 
\end{centering}

\end{frame} 


\begin{frame}{Fast Transform - continued}

\noindent - Start at finest scale and calculate differences and averages
\vskip .01in
\noindent - Use Averages at next coarser scale to get new set of differences
($b_{j,k}$) and averages ($a_{j,k}$)
\vskip .05in
\noindent
And the coefficients are simply the differences ($b_{j,k}$) and the average for the
coarsest scale ($a_{0,0}$):
\[ \left[ \begin{array}{cccccccc} 
a_{0,0} & 
b_{0,0} &  
b_{1,0} &  
b_{1,1} &  
b_{2,0} &  
b_{2,1} &  
b_{2,2} &
b_{2,3} \\ 
\end{array} \right]
\]
\vskip .02in
\noindent
For Haar Wavelet:
\[ a_{j-1,k} = c_0 a_{j,k} + c_1 a_{j,k+1} \]
\[ b_{j-1,k} = c_0 a_{j,k} - c_1 a_{j,k+1} \]

\vskip .01in
\noindent - The differences are the coefficients at each scale. Averages used for next scale.
\vskip .01in
\noindent
{\bf $\bullet$ How do we store all this?} (eg, Numerical Recipes algorithm).

\end{frame} 

\begin{frame}{Intermittent Signal - wavelet transform} 


\vskip .01in
\begin{centering}
\begin{figure}[ht]
\begin{tabular}{cc}
\includegraphics[width=.3\textwidth]{figures/two_freq_partial.ps} & 
\includegraphics[width=.3\textwidth]{figures/wav_coef_intermit.ps} \\
{\bf Signal} & {\bf Transform} \\ 
\end{tabular}
\end{figure}
\end{centering}
\vskip .01in
\noindent
$\bullet$ Higher Frequency intermittent signal shows up in the $j=5,6$
scales, and only in the middle portion of the domain. Localized frequency data
is found.

\end{frame} 

\begin{frame}{Data Compression - Efficient Representation}

\begin{centering}
\begin{figure}[ht]
\begin{tabular}{cc}
\includegraphics[width=.28\textwidth]{figures/local_fun.ps} &
\includegraphics[width=.28\textwidth]{figures/local_fun_wav_16.ps} \\ 
{\bf Original Localized Data, 128 points} & {\bf 16 term Wavelet Reconstruction}\\
\includegraphics[width=.28\textwidth]{figures/f_recon_16.ps} &  \\
{\bf 16 term Fourier Reconstruction} & \\  
\end{tabular}
\end{figure}
\end{centering}
\vskip .01in
\noindent
 Localized function is represented with more accuracy using 16 wavelet coefficients because only
 significant coefficients need to be retained. Fourier
coefficients are global.

\end{frame} 


\begin{frame}{Denoising by soft thresholding} 

\noindent
Representation is spread to many (if not all) wavelet coefficients.
\vskip .02in
\noindent
 Noise coefficients are relatively small in amplitude.
\vskip .02in
\noindent
$\bullet$ How do we remove the noise contribution in the coefficients?
\vskip .02in
noindent
Ans: Shrink {\bf all} of them just a little.
\vskip .02in
\noindent
$\bullet$ How much do we shrink each coefficient?
\vskip .02in
\noindent
$t = \sqrt{2 log(n)} \sigma / \sqrt{n} $
\vskip .02in
\noindent
where $n$= number of data points and $\sigma$=noise standard deviation, and t is the amount to shrink the coefficient towards 0.

\end{frame} 

\begin{frame}{Example} 

\begin{centering}
\begin{figure}[ht]
\begin{tabular}{cc}
\includegraphics[width=.4\textwidth]{figures/noisy.ps} &
\includegraphics[width=.4\textwidth]{figures/denoised.ps} \\ 
{\bf Noisy Signal} &{\bf Wavelet coefficients Shrunk toward zero.} \\
\end{tabular}
\end{figure}
\end{centering}
\vskip .02in
\noindent
$\bullet$ Soft Thresholding takes advantage of the fact that white noise is represented equally by all coefficients.
\vskip .02in
\noindent
$\bullet$ Data compression means that coherent signal is represented by just a few coefficients with relatively
large values. White noise (or non-white as well) is spread over many coefficients, adding just a small amount to the magnitude of each.

\end{frame} 

\begin{frame}{The Continuous Morlet Wavelet Transform} 

The Morlet mother wavelet is a complex exponential (Fourier) with a Gaussian envelop which
ensures localization:
\vskip .01in
\[ \psi(t) = exp(i \omega_0 t) exp(-t^2/2\sigma^2) \]
where $\omega_0$ is the frequency and $\sigma$ is a measure of the spread or support.
\vskip .01in
\noindent
Note that while the footprint is {\bf infinite}, the exponential decay creates an effective footprint
which is relatively compact.
\vskip .01in
\noindent
Translations and dilations of the Morlet wavelet:
\[
\psi_{(b,a)}(t) = \frac{1}{a}exp\left[i \omega_0 \left(\frac{t-b}{a}\right)\right]exp\left[-\left(\frac{t-b}{a}\right)^2/2\sigma^2\right]
\]
\noindent
{\bf The Morlet wavelet using matlab}
\vskip .01in
In matlab, the Morlet mother wavelet can be constructed using the command:
\vskip .01in
\noindent
[psi,x] =  Morlet(-8,8,128);
\vskip .01in
\noindent
on 128 grid points, and domain of [-8,8].

\end{frame}

\begin{frame}{Morlet - continued}

\begin{centering}
\begin{figure}[htp]
\begin{tabular}{c}
\includegraphics[width=.5\textwidth]{figures/morlet.eps} \\ 
\end{tabular}
\end{figure}
\end{centering}

\end{frame}


\begin{frame}{Morlet transform of a multi-scale sine wave}

Given a time series:
\vskip .01in
$ y(t) = sin(2\pi t) + sin(32\pi t)$    when  $.3 \leq t \leq .6 $ 
\vskip .01in
$ sin(2\pi t)$  otherwise  
\vskip .01in
on 128 grid points,
 the Morlet transform can be calculated and plotted using the command:
\vskip .02in
\noindent
coef=cwt(y,[1 2 4 8 16 32 64 128],'morl','plot')
\vskip .02in
\noindent
$\bullet$ the top level of the coefficient plot shows bright (or large) values for scale 128 (largest scale).
\vskip .02in
\noindent
$\bullet$ The next two layers are not zero, indicating {\it leakage} between different time scales.
\vskip .01in
\begin{centering}
\begin{figure}[htp]
\begin{tabular}{cc}
\includegraphics[width=.3\textwidth]{figures/sine_function.eps} &
\includegraphics[width=.3\textwidth]{figures/morlet_coef.eps} \\
\end{tabular}
\end{figure}
\end{centering}

\end{frame} 

\begin{frame}{Example: Southern Oscillation Index Time Series 1951-2005}
The SOI is the monthly pressure fluctuations in air pressure between Tahiti and Darwin. It is
generally a noisy time series and benefits from some smoothing. Soft thresholding is a way to
remove the noise in the signal without removing important information. The multivariate ENSO signal
comes from sea level pressure, surface wind, sea surface tempemperature, surface air temperature and
total amount of cloudiness.
\begin{figure}[ht]
\begin{centering}
\begin{tabular}{cc}
\includegraphics[width=.25\textwidth]{figures/original_soi_signal.eps} & 
\includegraphics[width=.25\textwidth]{figures/denoised_soi_signal.eps}\\
{\bf Multivariate ENSO Signal from CDC}\\
\includegraphics[width=.25\textwidth]{figures/ts.ps} \\
\end{tabular}
\end{centering}
\end{figure}

\end{frame}




\end{document}

